\documentclass[border=20pt]{standalone}
\renewcommand\familydefault{\sfdefault} % Default family: serif 
%\usepackage[usenames,dvipsnames]{xcolor}
\usepackage[x11names]{xcolor}
\usepackage{tikz}
\usepackage{soulutf8}
\usetikzlibrary{arrows,fit,positioning,shapes,calc}

%\definecolor{WIRE}{HTML}{002FA7} % Klein Blue
\usepackage[normalem]{ulem}

\tikzstyle{every entity} = []
\tikzstyle{every weak entity} = []
\tikzstyle{every attribute} = []
\tikzstyle{every relationship} = []
\tikzstyle{every link} = []
\tikzstyle{every isa} = []

\tikzstyle{link} = [>=triangle 60, draw, thick, every link]

\tikzstyle{total} = [link, double, double distance=3pt]

\tikzstyle{entity} = [rectangle, draw, black, very thick,
minimum width=6em, minimum height=3em,
every entity]

\tikzstyle{weak entity} = [entity, double, double distance=2pt,
every weak entity]

\tikzstyle{attribute} = [ellipse, draw, black, very thick,
minimum width=5em, minimum height=2em,
every attribute]

%\tikzstyle{key attribute} = [attribute, font=\bfseries]

\tikzstyle{multi attribute} = [attribute, double, double distance=2pt]

\tikzstyle{derived attribute} = [attribute, dashed]

%\tikzstyle{discriminator} = [attribute, font=\itshape]

\tikzstyle{relationship} = [diamond, draw, black, very thick,
minimum width=2em, aspect=1,
every relationship]

\tikzstyle{ident relationship} = [relationship, double, double distance=2pt]

\tikzstyle{isa} = [isosceles triangle, isosceles triangle apex angle=60,
shape border rotate=-90,
draw, black, very thick, minimum size=3em,
every isa]

% for text un key attributes
\newcommand{\key}[1]{\underline{#1}}
\newcommand{\pkey}[1]{\dashuline{#1}}

% for text in discriminator attributes
\def\discriminator{\bgroup 
	\ifdim\ULdepth=\maxdimen  % Set depth based on font, if not set already
	\settodepth\ULdepth{(j}\advance\ULdepth.4pt\fi
	\markoverwith{\kern.15em
		\vtop{\kern\ULdepth \hrule width .3em}%
		\kern.15em}\ULon}

%%


\definecolor{MediumPurple1}{rgb}{0.58, 0.44, 0.86}
\definecolor{Chartreuse2}{rgb}{0.5, 1.0, 0.0}
\tikzset{every entity/.style={draw=orange, fill=orange!20}}
\tikzset{every attribute/.style={draw=MediumPurple1, fill=MediumPurple1!20}}
\tikzset{every relationship/.style={draw=Chartreuse2, fill=Chartreuse2!20}}

%% Variable for participation constraint
\newcommand{\cM}{\mathrm{M}}
\newcommand{\cN}{\mathrm{N}}
\newcommand{\cO}{\mathrm{O}}
\newcommand{\cP}{\mathrm{P}}

\begin{document}


\Frame(0,0){1}[PKLOYEE]{
	Fname/A,
	Minit/A,
	Lname/A,
	Ssn/PK,
	Bdate/A,
	Address/A,
	Sex/A,
	Salary/A,
	Super-Ssn/A,
	Dno/A};

\Frame(0,-2.5){2}[DEPARTMENT]{
	Dname/A,
	Dnumber/PK,
	Mgr-ssn/A,
	Mgr-Start-date/A};

\Frame(0,-5){3}[DEPT-LOCATIONS]{
	Dnumber/PK,
	Dlocations/PK};

\Frame(0,-7.5){4}[PROJECT]{
	Pname/A,
	Pnumber/PK,
	Plocation/A,
	Dnum/A};

\Frame(0,-10){5}[WORKS ON]{
	Essn/PK,
	Pno/PK,
	Hours/A};

\Frame(0,-12.5){6}[DEPENDENT]{
	Essn/PK,
	Dependent-Name/A,
	Sex/A,
	Bdate/A,
	Relationship/A};

\draw[FK] % From Essn6 to Ssn1  
(Ssn1)++(0.1,0) -- ++(0,-.55) -- ++(4.5,0) coordinate (inter) %inter is the name of coordinate register
-- (Essn6 -| inter) -- ++(0,-0.4) coordinate (inter)  % to calculate intersections.
-- (Essn6 |- inter) --++(0,0.4); %
%Essn -- Ssn id 5
\draw[FK]
(Ssn1)++(-0.1,0) -- ++(0,-.7) -- ++(4.55,0) coordinate (inter) %some shift using (Ssn1)++(shiftx,shifty)
-- (Essn5 -| inter) -- ++(0,-0.4) coordinate (inter)
-- (Essn5 |- inter) --++(0,0.4); %
\draw[FK]
(Pnumber4) -- ++(0,-.5) -- ++(1,0) coordinate (inter)
-- (Pno5 -| inter) -- ++(0,-0.2) coordinate (inter)
-- (Pno5 |- inter) --++(0,0.2); %

\draw[FK]
(Dnumber2) -- ++(0,-.75) -- ++(4,0) coordinate (inter)
-- (Dnum4 -| inter) -- ++(0,-0.2) coordinate (inter)
-- (Dnum4|- inter) --++(0,0.2); %

\draw[FK]
(Dnumber2)++(-.2,0) -- ++(0,-.9) -- ++(1.75,0) coordinate (inter)
-- (Dnumber3 -| inter) -- ++(0,-0.2) coordinate (inter)
-- (Dnumber3 |- inter) --++(0,0.2); %

\draw[FK]
(Ssn1)++(-0.3,0) -- ++(0,-0.85) -- ++(3.5,0) coordinate (inter)
-- (Mgr-ssn2 -| inter) -- ++(0,-0.2) coordinate (inter)
-- (Mgr-ssn2 |- inter) --++(0,0.3); %

\draw[FK]
(Dnumber2)++(0.2,0) -- ++(0,-.6) coordinate (inter) -- (Dno1 |- inter) -- (Dno1); %

\draw[FK]
(Ssn1)++(0.3,0) -- ++(0,-.4) coordinate (inter) -- (Super-Ssn1 |- inter) -- (Super-Ssn1); %

\end{tikzpicture}
\end{document}