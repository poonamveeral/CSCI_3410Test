\documentclass[border=20pt]{standalone}
\renewcommand\familydefault{\sfdefault} % Default family: serif 
%\usepackage[usenames,dvipsnames]{xcolor}
\usepackage[x11names]{xcolor}
\usepackage{tikz}
\usepackage{soulutf8}
\usetikzlibrary{arrows,fit,positioning,shapes,calc}

%\definecolor{WIRE}{HTML}{002FA7} % Klein Blue
\usepackage[normalem]{ulem}

\tikzstyle{every entity} = []
\tikzstyle{every weak entity} = []
\tikzstyle{every attribute} = []
\tikzstyle{every relationship} = []
\tikzstyle{every link} = []
\tikzstyle{every isa} = []

\tikzstyle{link} = [>=triangle 60, draw, thick, every link]

\tikzstyle{total} = [link, double, double distance=3pt]

\tikzstyle{entity} = [rectangle, draw, black, very thick,
minimum width=6em, minimum height=3em,
every entity]

\tikzstyle{weak entity} = [entity, double, double distance=2pt,
every weak entity]

\tikzstyle{attribute} = [ellipse, draw, black, very thick,
minimum width=5em, minimum height=2em,
every attribute]

%\tikzstyle{key attribute} = [attribute, font=\bfseries]

\tikzstyle{multi attribute} = [attribute, double, double distance=2pt]

\tikzstyle{derived attribute} = [attribute, dashed]

%\tikzstyle{discriminator} = [attribute, font=\itshape]

\tikzstyle{relationship} = [diamond, draw, black, very thick,
minimum width=2em, aspect=1,
every relationship]

\tikzstyle{ident relationship} = [relationship, double, double distance=2pt]

\tikzstyle{isa} = [isosceles triangle, isosceles triangle apex angle=60,
shape border rotate=-90,
draw, black, very thick, minimum size=3em,
every isa]

% for text un key attributes
\newcommand{\key}[1]{\underline{#1}}
\newcommand{\pkey}[1]{\dashuline{#1}}

% for text in discriminator attributes
\def\discriminator{\bgroup 
	\ifdim\ULdepth=\maxdimen  % Set depth based on font, if not set already
	\settodepth\ULdepth{(j}\advance\ULdepth.4pt\fi
	\markoverwith{\kern.15em
		\vtop{\kern\ULdepth \hrule width .3em}%
		\kern.15em}\ULon}

%%


\definecolor{MediumPurple1}{rgb}{0.58, 0.44, 0.86}
\definecolor{Chartreuse2}{rgb}{0.5, 1.0, 0.0}
\tikzset{every entity/.style={draw=orange, fill=orange!20}}
\tikzset{every attribute/.style={draw=MediumPurple1, fill=MediumPurple1!20}}
\tikzset{every relationship/.style={draw=Chartreuse2, fill=Chartreuse2!20}}

%% Variable for participation constraint
\newcommand{\cM}{\mathrm{M}}
\newcommand{\cN}{\mathrm{N}}
\newcommand{\cO}{\mathrm{O}}
\newcommand{\cP}{\mathrm{P}}

\begin{document}



% ACTOR(Id (PK), Name, Birthdate)
% MOVIE(Title (PK), Year, Length, Studio)
% THEATER(Name (PK), Street, City, State, Zip)
% ACTING(ActorId (PK, FK to ACTOR.Id), MovieTitle (PK, FK to MOVIE.Title))
% SHOWING(TheaterName (PK, FK to THEATER.Name), MovieTitle(PK, FK to MOVIE.Title), Day (PK), Time (PK))

\Frame(0,0){1}[CUSTOMER]{
	Name/PK,
	Phone/A,
	AssignedEmployee/A,
	LevelOfSatisfaction/A};

\Frame(0,-2.5){2}[BIKE]{
	DroppedTimeDate/PK,
	Customer/PK,
	Color/A,
	Brand/A,
	RepairedBy/A};

\Frame(10,0){3}[EMPLOYEE]{
	Name/PK};

\Frame(10,-2.5){4}[SPECIALTY]{
	Employee/PK,
	Specialty/PK};

\draw[FK] % From CUSTOMER.AssignedEmployee to EMPLOYEE.Name
(Name3)++(-0.2,0)  -- ++(0,-.55)  coordinate (inter)
-- (AssignedEmployee1 |- inter) --++(0, 0.5);

\draw[FK] % From BIKE.Customer to CUSTOMER.Name
(Name1)++(0.1,0) -- ++(0,-0.5)  --++(-1.5, 0)
 --++(0, -3)
coordinate (inter)
-- (Customer2 |- inter) --++(0, 1);

\draw[FK] % From BIKE.RepairedBy to EMPLOYEE.Name
(Name3)++(0,0) -- ++(0,-.75)--++(-1, 0) --++(0, -3)
coordinate (inter)
-- (RepairedBy2 |- inter) --++(0, 1.3);

\draw[FK] % From SPECIALTY.Employee to EMPLOYEE.Name
(Name3)++(0.2,0) -- ++(0,-.85)--++(-1, 0) --++(0, -2.9)
coordinate (inter)
-- (Employee4 |- inter) --++(0, 1.25);
\end{tikzpicture}
\end{document}